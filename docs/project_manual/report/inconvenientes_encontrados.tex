\begin{itemize}
    \item El primer inconveniente encontrado fue a la segunda semana de
    proyecto, cuando ya se habían implementado algunas de las animaciones del
    trabajo, nos encontramos con que la librería SDL1.2 tiene de por si perdidas
    de memoria de las cuales valgrind nos advierte en cada ejecución. Si bien
    los leaks no aumentan conforme el juego transcurre, si están presentes y son
    inevitables.
    \item En segundo lugar, el trabajar con un integrante menos durante todo el
    desarrollo hizo que tuviéramos que trabajar un poco mas en algunas partes
    del proyecto que no podían faltar en la entrega y , en consecuencia, nos
    atrasamos con el cronograma establecido.
    \item Por otro lado, al no haber trabajado nunca con una librería gráfica,
    todas las semanas era necesario aprender algo nuevo y, lamentablemente,
    no hay mucha documentación de la librería \code{SDL1.2} que sea útil,
    aunque si la hay de \code{SDL2.0}.
    \item Nos costó mucho sincronizar al cliente y al servidor para que las
    animaciones cobraran una fluidez aceptable para un jugador. A tal punto que
    algunas cosas se pueden mejorar todavía en ese aspecto.
    \item A la hora de serializar la información nos tomo mucho tiempo definir
    el formato en la que se enviaría la información por sockets y aun habiéndolo
    hecho tuvimos que realizar cambios.
\end{itemize}