\hfill \break
\underline{\textbf{Semana 1 a 2:}}
Durante las primeras 3 semanas cada uno trabajó por separado,
dado que aún no había nada para ensamblar. Del lado del servidor estas semanas
se utilizaron para la implementación del algoritmo A-star y a realizar numerosas
pruebas en el mismo para el posterior ensamble. Mientras que del lado del cliente
estas semanas se utilizaron para decidir que librería gráfica se iba a utilizar
(decidimos utilizar SDL), a la configuración de la misma y a la creación de las
primeras animaciones de movimiento con algunas unidades.
A esta altura del proyecto se habían cumplido con los tiempos del cronograma
recomendado por la cátedra dado que el mismo pedía algunas animaciones y la
implementación y correcto funcionamiento del algoritmo A-star.

\hfill \break
\underline{\textbf{Semana 2 a 4:}}
En este periodo comenzamos a juntar las distintas partes del trabajo
en un único módulo (cliente y servidor juntos) para ir probando las
funcionalidades de movimiento. Al hacerlo nos encontramos con las primeras
dificultades, dado que había que sincronizar correctamente las acciones del
cliente y el server para que la animación se viera bien. \\
Dada la baja de uno de los integrantes de nuestro equipo, del lado del cliente
hubo que reestructurar el cronograma para que ciertas partes que eran necesarias
por el servidor estuvieran presentes. Por esto se decidió retrasar la sección
del movimiento de la cámara a la semana 5 y se priorizo la implementación de las
interfaces de las fábricas para la creación de unidades, así como la selección
de las mismas vía mouse y las animaciones y acciones de disparo y movimiento.
Por el lado del servidor se implementaron la lógica de los disparos, monitoreo,
creación de unidades y conquista de territorios. A esta altura comenzamos a
retrasarnos un poco del lado del cliente, dado que hubo que realizar la
implementación de interfaces que habían quedado pendientes del tercer integrante
en las 2 primeras semanas. El cronograma estipulaba que en este momento
debíamos tener: \\

\underline{\textbf{Servidor:}} Lógica del modelo sobre la creación de unidades,
capturas de vehículos, ataque entre unidades.\\

\underline{\textbf{Cliente:}} movimiento de cámara, selección de unidades,
selección de qué unidad crear, acciones de atacar. \\

Dado que se hicieron las interfaces de las fábricas, la parte del movimiento de
cámara aun no se había completado en esta etapa.

\hfill \break
\underline{\textbf{Semana 4 a 6:}}
Comenzamos a juntarnos con mayor frecuencia, dado que a medida que las
funcionalidades estaban terminadas de ambos lados del proyecto, era necesario
probarlas para ver la correcta coordinación del cliente y el servidor. A esta
altura no teníamos un cliente y un servidor por separados, sino que probábamos
las funcionalidades en un solo módulo en el que ensamblamos ambas partes.

Del lado del cliente se implementaron el scroll del mapa y la cámara que habían
quedado pendientes en semanas anteriores y la extensión de todo lo hecho
anteriormente a todas las unidades, tipos de tile y edificios (dado que por
practicidad las pruebas las hacíamos con 2 tipos de unidades, dado que la
extensión, si bien tomaba tiempo, era simplemente hacer cambios chicos).

Del lado del servidor se terminaron de implementar la lógica de victoria y
derrota.

A esta altura del cronograma ambas partes debían tener terminada toda la lógica.
Sin embargo del lado del cliente quedaban pendientes las explosiones y la
implementación de la lógica para moderar el acceso de las interfaces de las
fábricas a los distintos jugadores.

\hfill \break
\underline{\textbf{Semana 1 a 2:}}
En este periodo separamos el cliente el servidor para que la comunicación de los
mismos fuera a través de sockets. Previo a esto fue necesario definir un
protocolo de comunicación y de qué forma haríamos para darle un formato fácil de
parsear a la información que iría desde el servidor al cliente. Una vez hecho
esto se procedió a la inclusión de sockets.

Luego de testeada la comunicación a través de sockets se comenzó con la parte de
concurrencia de ambos lados, agregamos los hilos en aquellas partes consideramos
que mejorarian la eficiencia de la aplicación y colocamos los locks en las zonas
críticas.

En paralelo a la corrección de bugs de race conditions provocados por la
concurrencia redactamos los informes para la presentación final.