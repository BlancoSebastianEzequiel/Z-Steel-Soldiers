\subsection{Login}
    Luego de ejecutar el servidor y ejecutar un cliente donde en nuestro caso
    particular se deben abrir cuatro consolas (1 server y 3 clientes), se elije
    el número de equipo, donde se puede jugar en modo todos contra todos si cada
    uno de los clientes se loguea en un número de equipo distinto o aliarse por
    ejemplo dos contra dos si dos clientes se loguean en un mismo número de equipo
    \def\path{Login.jpg}
    \def\scale{.6}
    \def\text{Login}
    \input{report/image.tex}
\subsection{Comienzo de partida}
    Luego de realizado el login se abrirá el mapa donde transcurrirá la partida.
    Cada cliente comienza con un robot Grunt.
    \def\path{Inicio.png}
    \def\scale{.6}
    \def\text{inicio}
    \input{report/image.tex}
    El usuario podrá entonces comenzar a desplazarse con el mouse o las flechas
    del teclado. Podrá seleccionar sus unidades y sus edificios.
\subsection{Selección de unidades y edificios}
    Para seleccionar una unidad o edificio se debe posicionar el mouse sobre los
    mismos y hacer click izquierdo. En el caso de las unidades, no se notará ningún
    cambio o mensaje, por el lado de los edificios se abrirá la interfaz de creación
    de unidades.
    \def\path{Seleccion.png}
    \def\scale{.6}
    \def\text{seleccion}
    \input{report/image.tex}
\subsection{Movimiento de unidades}
    Luego de haber seleccionado una unidad, como fue detallada en la sección
    anterior , se debe hacer click derecho con el mouse en la posición donde se
    desea que la unidad se desplace. En caso de seleccionar un lugar inaccesible
    donde por ejemplo haya una isla de pasto bloqueada por lava, la unidad no se
    moverá.
    \def\path{Movimiento.png}
    \def\scale{.6}
    \def\text{Movimiento}
    \input{report/image.tex}
\subsection{Ataque}
    Una vez seleccionada una unidad, si se desea atacar a una unidad enemiga u
    edificio enemigo, se procederá a hacer click derecho sobre el mismo. Luego de
    esto la unidad se desplazará hasta encontrarse en rango de disparo en caso de
    ser necesario y disparará.
    \def\path{Disparo1.png}
    \def\scale{.6}
    \def\text{Disparo}
    \input{report/image.tex}

    \def\path{DisparoEdificio.png}
    \def\scale{.6}
    \def\text{Disparo a edificio}
    \input{report/image.tex}
\subsection{Victoria y derrota}
    Para ganar una partida se deben destruir todos los fuertes enemigos, o
    unidades o captura de territorios. En cualquiera de los tres casos aparecerá los
    siguientes mensajes en pantalla.
    \def\path{Ganaste.png}
    \def\scale{.6}
    \def\text{Ganaste}
    \input{report/image.tex}

    \def\path{Perdiste.png}
    \def\scale{.6}
    \def\text{Perdiste}
    \input{report/image.tex}