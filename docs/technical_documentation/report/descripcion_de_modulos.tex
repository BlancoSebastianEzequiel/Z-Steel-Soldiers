\subsection{Cliente}
    \subsubsection{Modulo de dibujo}
        Como se dijo anteriormente, la responsabilidad de este módulo es la de
        convertir la información del modelo, la cual se encuentra en el server,
        en información visual que un usuario pueda entender. Las clases que más
        se destacan en este módulo son \code{Map} y \code{GameObject}. \\
        A continuación se detallan las clases que componen este módulo, así como
        sus responsabilidades:
        \paragraph{Map}
            Es la clase con mayor lógica de este módulo. Almacena todo el
            conjunto de unidades, edificios, objetos de terreno, balas ,
            explosiones y tiles presentes en el mapa que se dibuja. Su mayor
            responsabilidad es dibujar en cada iteración del cliente todos los
            objetos anteriormente mencionados. Para realizar esto, posee un
            método \code{draw} que recibe en cada iteración un objeto de tipo
            \code{SDL\_Surface*}, que es la pantalla donde se dibujara la escena.
            El mapa entonces recorre los elementos de la escena uno por uno y
            les comunica que deben dibujarse, luego cada objeto en particular
            es responsable de saber cómo dibujarse con el sprite indicado. Los
            métodos más importantes de esta clase se detallan a continuación:
            \def\path{client/Map.jpg}
            \def\scale{.6}
            \def\text{Map}
            \begin{figure}[!htbp]
    \begin{adjustbox}{addcode={
        \begin{minipage}{\width}}{
            \caption{\text}
        \end{minipage}},rotate=360,center}
        \includegraphics[scale=\scale]{\path}
    \end{adjustbox}
\end{figure}
\FloatBarrier
            \subparagraph{draw}
                Este metdoo recibe como parametro la pantalla donde se dibujara la
                escena, y recorre todas las estructuras de datos donde se almacenan
                los objetos y le pide a cada objeto en particular que se dibuje.
                Adicionalmente en este método se realizan la liberación de recursos
                de aquellos objetos que, por alguna razón, ya no deben ser
                dibujados, ya sea una unidad que murió, una bala que ya colisionó o
                bien una explosión que ya terminó su animación. Como ya no es
                necesario dibujar estos objetos se realizan en métodos aparte la
                liberación de recursos de los mismos.
            \subparagraph{create\_grunt}
                Simplemente agrega un nuevo robot de tipo grunt al mapa de unidades,
                para que en el siguiente llamado al método \code{draw}, este
                aparezca en la escena. Existen métodos similares a este como
                \code{create\_laser}, \code{create\_pyro}, \code{create\_light\_tank},
                etc. Todos son análogos a este con la diferencia de que tipo de
                unidad agregan a la escena.
            \subparagraph{unit\_move}
                Recibe el id de la unidad que se tiene que mover, las coordenadas
                del destino y la velocidad con la que se debe desplazar. Este método
                le asigna a dicha unidad una acción \code{ClientWalkAction}, dicha
                acción en cada llamada al método \code{draw} hará que la unidad se
                desplace una parte del trayecto que debe recorrer hasta llegar a la
                posición final. En resumen lo que este método hace es setear a una
                unidad una acción de movimiento que no solo hace que esta se
                desplace, sino que también se encarga de que la unidad cambie el
                sprite de movimiento a medida que cambia su posición.\\
                Analogamente se enceuntra el metodo \code{bullet\_move}, que
                funciona de fomra analoga a este, pero para las balas
            \subparagraph{unit\_shoot}
                Este método recibe por parámetro el id de la unidad que ejecuta un
                disparo y además un objeto target (que puede ser otra unidad, un
                edificio o bien una roca o puente). Su función es asignar a la
                unidad especificada por id una acción \code{ClientShootAction}, la
                cual en cada ciclo de draw hará que la unidad cambie su sprite al
                de uno de disparo. Además lo que hace es crear una bala en la
                posición de la unidad que dispara y la agrega al mapa de proyectiles,
                para que en la siguiente llamada al método draw esta aparezca y
                comience a ejecutar su movimiento hacia el objetivo.
        \paragraph{GameObject}
            Una gran parte de los objetos en el cliente son simplemente
            entidades que tienen una posición en un determinado instante,
            representada por las coordenadas x e y, y además un id para
            identificarlos unívocamente. A raíz de esto la clase
            \code{GameObject} intenta generalizar el manejo de las entidades
            ofreciendo una interfaz que es básica para todo objeto que se quiera
            dibujar en el mapa. La razón por la cual esta clase es tan importante
            en el modulo de dibujo es por que es la clase responsable de
            efectivamente dibujar el objeto que representa en la pantalla cuando
            el mapa así se lo pide. Entre sus métodos mas destacados se
            encuentran:
            \def\path{client/GameObject.jpg}
            \def\scale{.6}
            \def\text{GameObject}
            \begin{figure}[!htbp]
    \begin{adjustbox}{addcode={
        \begin{minipage}{\width}}{
            \caption{\text}
        \end{minipage}},rotate=360,center}
        \includegraphics[scale=\scale]{\path}
    \end{adjustbox}
\end{figure}
\FloatBarrier
            \subparagraph{get\_x/get\_y}
                Dado que todo objeto dibujable debe tener una posición,
                \code{gameObject} permite conocerla mediante estos 2 métodos.
            \subparagraph{draw}
                Se trata del método más importante de la clase, el mismo se
                encarga de que el fotograma del objeto a mostrar en esta
                renderizacion sea el que efectivamente aparezca en la escena
                para que el usuario lo vea.
            \subparagraph{is\_selected}
                es un método más relacionado con la interacción usuario-aplicación,
                pero tiene cierta relación con el dibujo. Todo objeto debe ser
                capaz de saber si el mouse lo está seleccionando o no, según la
                posición del mapa donde se encuentre dibujado. Este método cumple
                con la tarea de informar a quien le invoque si el objeto en cuestión
                está seleccionado o no.\\
                En particular las clases \code{ClientBuilding}, \code{Unit},
                \code{ClientAmmo}, \code{ClientExplosion} y \code{TerrainOject}
                heredan de \code{GameObject} y todas implementan por diferencia el
                método \code{draw}. Básicamente en cada método \code{draw} de las
                clases hijas se selecciona que fotograma dibujara el \code{draw} de
                \code{GameObject}.
        \paragraph{Frame}
            Una de las clases más primitivas del modulo de dibujo ya que tiene
            muy poco comportamiento. En resumen cada objeto que tenga que
            animarse debe poseer una serie de fotogramas que sean cambiados en
            cada dibujo de la escena. Esta clase representa dichos fotogramas,
            resultando que cada \code{GameObject} esté relacionado con 1 o más.
            \subparagraph{get\_image}
                Como un fotograma no es más que una imagen, este método simplemente
                devuelve la imagen asociada al fotograma, en forma de una
                \code{SDL\_Surface*} (tipo de objeto que es dibujable), para que otro
                objeto lo dibuje en una escena.
    \subsubsection{Módulo de interacción con el usuario}
        Este módulo se encuentra compuesto por dos clases que se encargan de que
        el usuario pueda influir en el desarrollo del juego, dichas clases son
        \code{ClientMouseEventHandler} y \code{ClientKeyboardEventHandler}.
        \def\path{client/handlers.jpg}
        \def\scale{.6}
        \def\text{handlers}
        \begin{figure}[!htbp]
    \begin{adjustbox}{addcode={
        \begin{minipage}{\width}}{
            \caption{\text}
        \end{minipage}},rotate=360,center}
        \includegraphics[scale=\scale]{\path}
    \end{adjustbox}
\end{figure}
\FloatBarrier
        \paragraph{ClientMouseEventHandler}
            Permite la interacción entre el usuario y el juego mediante el mouse.
            El mismo permite a un jugador seleccionar unidades, indicarles un
            lugar a donde se tienen que desplazar, ordenar el ataque a otra
            unidad u objeto, uso de interfaces de los edificios para la creación
            de unidades y además permite llevar a cabo un scroll de pantalla que
            se detallara en otro módulo. Los métodos más importantes de esta
            clase son:
            \subparagraph{handle\_left\_click}
                Este método se encarga única y exclusivamente de gestionar la
                selección de objetos en el juego. Cuando un usuario hace click
                con el botón izquierdo del mouse sobre el mapa, se almacena una
                posición en el \code{MouseEventHandler} la cual se contrasta
                contra la de todos los objetos del juego para saber si alguno
                fue seleccionado. En esencia este método no modifica el
                desarrollo del juego, pero permite que un jugador acceda al
                control de una de sus unidades o a alguna de las interfaces de
                creación de alguno de sus edificios.
            \subparagraph{handle\_right\_click}
                A diferencia del \code{handle\_left\_click}, este método si tiene
                un impacto en el desarrollo de una partida. Siempre y cuando una
                unidad propia haya sido seleccionada previamente mediante el
                click izquierdo, este método decidirá si en la posición en la
                que se hizo click derecho corresponde que dicha unidad se
                desplace o ataque a otro objeto/unidad presente en ese lugar.
            \subparagraph{handle\_mouse\_motion}
                Si bien este metodo esta muy relacionado con lo que se dibuja en
                la pantalla, consideramos que sigue siendo parte del módulo de
                interacción. Este método simplemente compara la posición
                anterior del mouse (previa al último desplazamiento del mismo)
                con la final de desplazamiento, y le notifica a la cámara del
                movimiento en caso de ser necesario un scroll en la pantalla.
        \paragraph{ClientKeyboardEventHandler}
            Esta clase está dedicada a permitir la
            interacción entre el usuario el juego mediante el teclado. A
            diferencia del \code{MouseEventHandler} este administrador se
            limita únicamente al movimiento de la cámara. Su método más
            importante es el siguiente:
            \subparagraph{handle\_event}
                Independientemente de la posicion de la camara, si el usuario
                aprieta alguna de las flechas del teclado este método hará que
                la cámara ejecute un scroll en la dirección de la flecha que se
                apretó. El teclado permite únicamente interacción con la cámara.
        \paragraph{ClientCamera}
            Entidad que muestra al jugador cierta porción del mapa en la
            cual trascurre el juego. El jugador puede desplazarse mediante el
            teclado y/o mouse, para que esta le muestre distintas partes del
            terreno según este lo requiera. Sus métodos más importantes son:
            \subparagraph{move\_x/move\_y}
                Estos métodos están relacionados con el movimiento del mouse
                sobre el mapa del juego. Cuando el mouse se encuentre en alguna
                posición de la pantalla cercana a los bordes y se mueva un poco
                más allá de cierto límite, se ejecutará un scroll en la
                dirección del desplazamiento, permitiendo al usuario ver otra
                porción del mapa(siempre y cuando no salga de los limites del
                mismo).
            \subparagraph{keyboard\_move\_x/keyboard\_move\_y}
                Estos dos métodos, aunque parecidos con los dos anteriores,
                están relacionados con el uso de las teclas de movimiento. A
                diferencia de move\_x y move\_y, estos métodos solo se aseguran de
                que no ocurra un scroll fuera del mapa, pero en cualquier otro
                caso correrán la cámara una cantidad de píxeles fija.
    \subsubsection{Módulo de comunicación cliente-servidor}
        Este módulo realiza algunas de sus tareas en paralelo respecto de los
        otros dos módulos, dada la necesidad de que la aplicación sea lo más
        eficiente posible. La información mostrada por pantalla del lado del
        cliente debe reflejar lo que ocurre del lado del servidor con la mayor
        exactitud posible. La importancia de este módulo radica en la necesidad
        de pedirle al servidor la información necesaria para que el módulo de
        dibujo la pueda representar visualmente. Entre las clases más
        importantes de este módulo se encuentran \code{ClientProxyGame},
        \code{ClientCommunicator} y \code{ClientInterpreter}.
        \def\path{client/Comunicacion.jpg}
        \def\scale{.6}
        \def\text{Communication}
        \begin{figure}[!htbp]
    \begin{adjustbox}{addcode={
        \begin{minipage}{\width}}{
            \caption{\text}
        \end{minipage}},rotate=360,center}
        \includegraphics[scale=\scale]{\path}
    \end{adjustbox}
\end{figure}
\FloatBarrier
        \paragraph{ClientProxyGame}
            Esta clase es un espejo de lo que ocurre en el juego del lado del
            servidor. Almacena un conjunto de objetos proxy que contienen la
            información necesaria para que el cliente pueda dibujar lo que
            ocurre en el server. Entre sus métodos más importantes se encuentran:
            \subparagraph{getModel}
                Arma un string en cierto formato, pidiéndole el modelo al
                servidor y lo encola en una cola de peticiones. Luego de un
                tiempo el comunicador desencolará esta petición y la enviará
                vía socket hacia el server.
            \subparagraph{update}
                De forma similar a get model, arma un string en cierto formato,
                pidiendo al servidor que actualice las posiciones de los objetos
                que viven en el modelo, ya sean balas o unidades moviéndose.
            \subparagraph{RefreshProxyGame}
                Elimina todos aquellos objetos, unidades o proyectiles que
                desaparecieron del lado del server ya sea porque fueron
                destruidos, murieron o colisionaron respectivamente.
        \paragraph{ClientCommunicator}
            Es el encargado de desencolar las peticiones encoladas por el
            \code{ClientProxyGame} y enviarlas al server para que este actualice
            el modelo o lo envié vía socket. Por cuestiones de eficiencia, el
            comunicador corre en un hilo aparte, de esta forma independizamos la
            comunicacion via socket, del dibujo de la escena. Entre los métodos
            importantes del communicator se encuentran:
            \subparagraph{sendMessage}
                Recibe por parámetro un string que contiene un comando parseado
                en un formato que el server puede entender. Este método
                simplemente envía vía socket dicho string.
            \subparagraph{ReceiveMEssage}
                Recibe cualquier mensaje enviado via socket desde el servidor y
                lo almacena hasta que el \code{ClientGameProxy} se la pida para
                actualizar su información.\\
                En resumen el \code{ClientCommunicator} encapsula el envío y
                recepción de mensajes via socket, para poder hacerlo en paralelo.
                El protocolo de comunicación que utilizamos consiste en convertir
                los bytes a ser enviados a formato big endian, luego se procede a
                enviar el largo del mensaje en un primer send, seguido de otro send
                con los datos.
        \paragraph{ClientInterpreter}
            Objeto del que se vale el \code{ClientProxyGame} para poder parsear
            los strings recibidos por el ClientCommunicator desde el server y de
            esta forma interpretarlos para reconstruir los objetos del modelo en
            el lado del cliente. Sus métodos son similares, con la diferencia de
            que cada uno se encarga de parsear tipos de objetos del modelo
            distintos. Por ejemplo:
            \subparagraph{deserializeObject}
                Parsea los distintos campos del objeto, su posicion, id, estado,
                nivel de tecnología si fuese un edificio y se los asigna a un
                nuevo objeto proxy del mismo tipo que almacena el server, con la
                diferencia de que el objeto proxy no tiene ningún comportamiento,
                simplemente esta para obtener su información.\\
                El resto de los métodos siguen con la misma lógica, reciben un
                string en un formato determinado, el interpreter los parsea y
                crea un objeto del tipo análogo al que tiene el server con la
                formación extraída de dicho string, para que el \code{proxyGame}
                lo agregue a su copia del modelo.
\subsection{Servidor}
    \subsubsection{Modulo de aceptación de clientes}
        \paragraph{mainLoop}
            Este método se encarga de aceptar las nuevas conexiones de los
            clientes a través del socket de la clase que representa este módulo.
            Usando el método del socket llamado \code{accept}, el cual es
            bloqueante, espera un nuevo cliente. Cuando esto sucede, crea un
            hilo de comunicación, lo inicia y vuelve a esperar una
            nueva conexión.
        \paragraph{stop}
            Este método se encarga de finalizar la ejecución de todo el servidor.
            Recorre todos los hilos \code{ProxyClient} que tiene
            almacenados en un vector y llama a un método que estos poseen
            también llamado \code{Stop} el cual los finaliza correctamente.
    \subsubsection{Modulo de comunicación con el cliente}
        \paragraph{run}
            Este método es el que ejecuta el hilo en cuestión Se trata de un
            loop en el cual la condición de corte es una variable booleana la
            cual se ve modificada cuando se llama al método \code{stop()}.
            En cada iteración se recibe un pedido del cliente y se analiza
            que es lo que quiere realizar. En caso de que el pedido sea
            \code{"GET\_MODEL"}, el cual significa que el cliente pide una
            serialización del modelo, entonces se sabe que inmediatamente se
            requiere responder enviando la serialización y como ultimo, el
            mensaje \code{“end”} para que el cliente sepa cuando dejar de
            recibir el mismo.
            Además hay dos tipo de pedido del modelo. El inicial consta del
            envío de todo el mapa, con sus territorios y objetos para que de
            esta manera, el cliente pueda dibujar el mapa. Luego, como todo lo
            referidos a \code{Tiles} y territorios, no se modifica, no es
            necesario enviarlos mas.\\
            El otro tipo de envío de modelo es el de las unidades, municiones y
            objetos, que de hecho son las únicas tres cosas que el cliente
            necesita conocer para actualizar el dibujo del mapa.\\
            Por último, en caso de recibir el pedido \code{“CREATE\_PLAYER”}, el
            \code{ProxyClient} sabe que debe responder inmediatamente al
            cliente con el id del nuevo jugador creado.\\
            En caso de recibir otro tipo de pedido, como por ejemplo, el de
            crear cierta unidad, se llama a un método del modelo, el cual se llama
            \code{receivePetition()}  que se encarga de cumplir con el mismo.
        \paragraph{SendCommand}
            Este método recibe un string como argumento y se encarga de enviar
            al cliente dicho mensaje. Primero envía el largo del mensaje para
            que el cliente sepa la cantidad de bytes a recibir. Luego envía el
            string.
        \paragraph{ReceiveCommand}
            Se encarga de recibir y almacenar en un string el menaje enviado por
            el cliente. Primero sabe que va a recibir cuatro bytes que
            representan el largo del mensaje para que luego almacene en un
            buffer el mismo.
        \paragraph{ReturnModel}
            Este método recibe un vector con el modelo serializado, y se encarga
            de enviara de a uno cada string de cada posición del vector usando
            el método \code{sendCommand()}.
    \subsubsection{Modulo del modelo servidor}
        \paragraph{CreateRobotGrunt}
            Este método recibe el id del edificio que pidió la creación de la
            unidad. De ese modo, esta función se encarga de crear una clase
            \code{ServerTaskCreateRobotGrunt} la cual se encarga de ejecutar en
            su método \code{execute()} la creación concreta de la unidad. Pero
            este método será ejecutado cuando el tiempo de espera, hasta que
            dicha creación se llevada acabo, haya pasado. Esto ocurre así ya
            que esta clase, que hereda \code{Task}, tiene un tiempo de
            espera que conoce y usando la librería de c++ de time chequea que
            dicho tiempo haya pasado. Este método createRobotGrunt es
            equivalente a todas las restantes unidades.
        \paragraph{MoveUntil}
            Este método recibe la posición en ‘x’ y en ‘y’ en forma de píxeles
            y el id de la unidad que se desea mover. Lo que hace es buscar y
            obtener a la unidad mediante el diccionario de unidades. Luego
            llama a uno de los métodos de la misma llamado \code{goTo()} que
            devuelve una cola de clases NodePath la cual almacena una
            posición, y la posición a la cual se está moviendo. Es decir,
            calcula con el algoritmo \code{Astar} y genera una cola donde
            devuelve el camino mínimo a través de estas clases. Luego, el método
            \code{MoveUnit} crea las tareas de movimiento \code{TaskMove}
            donde se le pasa la clase \code{NodePath}. Estas tareas se irán
            cumpliendo cuando si tiempo de espera se haya consumido, el cual
            depende de la velocidad dela unidad y además es el tiempo acumulado
            ya que se suma a los tiempos de las tareas anteriores a sí misma.
            Este tiempo de espera se le pasa como argumento en el constructor.
        \paragraph{AttackUnit}
            Este método recibe el id de la unidad que dispara y el id de la unidad
            que recibirá el disparo. Verifica que en caso de que el atacante
            esté o no en rango de disparo. En el caso de que no lo esté llama
            al método moveUntil el cual le genera todas las tareas para que la
            unidad se mueva hasta estar en rango. Luego crea la tarea
            \code{TaskShoot} la cual al ejecutarse la misma, cuando su
            tiempo de espera se consuma, crea la munición correspondiente a la
            unidad y genera sus tareas de movimiento pero en este caso para la
            munición donde el tiempo de espera de cada uno de estos depende de
            la velocidad de la munición y que cada tarea posee el tiempo
            acumulado respecto a la anterior. También crea la tarea
            \code{ServerAttackUnit} donde su tiempo de espera es mínimamente
            mayor a la ultima tarea de movimiento de la munición que es la que
            llega a destino y que cuando se ejecuta le saca vida a la unidad que
            recibirá el disparo.
        \paragraph{AttackObject}
            Este método es equivalente al método \code{attackUnit} con la diferencia
            que en vez de buscar el id de una unidad como objetivo, busca el id
            de un objeto en el mapa de objeto del mapa.
        \paragraph{Update}
            Esta función se encarga de actualizar el modelo. Se encarga de iterar
            todas las unidades y preguntar si esta murió. En ese caso la agrega
            a un diccionario de unidades muertas y continua con la siguiente
            unidad. Verifica si la unidad está parada sobre una bandera, donde
            en ese caso cambio de dueño a su favor. Luego se fija en la cola de
            prioridad de tareas a cumplir de la unidad y se fija se la tarea de
            mayor prioridad, que sería la que tiene menor tiempo de espera, es
            ejecutable. Es decir, si su tiempo de espera se consumió. En caso
            positivo, la desencola y la ejecuta modificando así el estado de la
            unidad. Luego monitorea el estado de la unidad donde en caso de estar
            quieta, verifica si hay enemigos en rango, si está atacando unidades
            u objetos, verifica si murieron. Repite el mismo proceso para las
            municiones donde refresca las tareas de las mismas y en caso de que
            la munición haya impactado la agrega a un diccionario de municiones
            obsoletas. Y para los objetos verifica si están rotos. En caso de ser
            piedras o bloques de hielo, lo agrega a un diccionario de objeto
            rotos. En caso de que un fuerte esté roto, muestra la bandera del
            territorio asociado al fuerte.
    \subsubsection{Modulo de serializacion del modelo}
        \paragraph{deserializePetition}
            Recibe un string que representa un pedido del cliente y lo parsea.
            Verifica que función del modelo desea llamar y la llama pasándole los
            argumentos correspondientes que parseó de la petición.
        \paragraph{InitialSerialize}
            Este método llama a métodos privados que serializan las dimensiones
            del mapa, a los nodos, los objetos, los territorios.
        \paragraph{Serialize}
            Este método llama a métodos privados que serializan las unidades, los
            objetos y las municiones.