El proyecto se puede dividir en dos módulos principales que son el cliente y
el servidor. Estos a su vez, pueden dividirse en submódulos que pasamos a
detallar a continuación.
\subsection{Cliente}
    El cliente está representado con la clase \code{Player}, la misma ejecuta un
    loop durante la ejecución del juego en el cual, en cada iteración, se
    realizan tareas de dibujo, interacción con el usuario y comunicación con el
    server, cada una siendo responsabilidad de un modulo distinto, mencionado a
    continuación.
    \subsubsection{Módulo de dibujo}
        En este módulo del cliente se encuentra toda la lógica requerida para
        poder trasladar lo que ocurre en el juego del lado del servidor a una
        escena que pueda ser entendida por un usuario que hace uso de la
        aplicación. En este modulo las clases a destacar son \code{Map} y
        \code{GameObject}, que cumplen casi todas las tareas que hacen posible
        esto y cuyas responsabilidades se detallaran más en la sección de
        módulos a continuación.
    \subsubsection{Módulo de interacción con el usuario}
        En este módulo, como dice el titulo, se encuentran todas las clases que
        permiten que el usuario interactúe y tenga incidencia en el desarrollo
        de una partida. Las clases que son encargadas de hacer posible la
        interacción son \code{ClientMouseEventHandler} y
        \code{ClientKeyboardEventHandler} que, respectivamente, habilitan el
        ingreso de información tanto por mouse como por teclado. Otra clase
        importante en este módulo es la clase cámara, la cual muestra al usuario
        toda la información visual que necesita.
    \subsubsection{Módulo de comunicación Cliente-Servidor}
        En este módulo se encuentran todas las clases que involucran el traspaso
        de información de cliente a servidor. En el mismo se destacan las clases
        \code{ClientProxyGame} y \code{ClientCommunicator}. Su función es
        recibir, deserializar y almacenar la información enviada desde el
        servidor de tal forma que el cliente pueda entenderla y manipularla
        correctamente.\\
        Además son los encargados de realizar peticiones al servidor, como el
        envío del modelo cuando sea necesario, entre otras.\\
        Tanto el módulo de dibujo como el de interacción con el usuario se
        ejecutan en el mismo hilo, mientras que toda la parte de comunicación
        con el servidor se realiza en paralelo (módulo de comunicación), para
        lograr una mayor eficiencia.
\subsection{Servidor}
    El servidor está representado por la clase \code{Game} la cual posee
    distintos métodos que modifican, actualizan y depuran el modelo del juego.
    El loop principal del servidor está representado por la clase
    \code{ProxyClient} que es un hilo que se encarga de establecer la
    comunicación con un cliente en particular.
    \subsubsection{Módulo de aceptación de clientes}
        Este módulo está representado por la clase \code{Server} la cual tiene
        un loop principal en donde en cada iteración espera hasta que un cliente
        establezca una conexión, siendo este loop bloqueante, ya que se utiliza
        la aceptación de sockets.
    \subsubsection{Módulo de comunicación con el cliente}
        Este módulo está representado por la clase \code{ProxyClient} la
        cual es un hilo que recibe pedidos de parte del cliente a través de la
        comunicación TCP usando el socket que aceptó el módulo anterior. Además
        se encarga de notificarle el pedido al modelo de manera tal que este
        último realice los cambios deseados por el cliente. Por último, se
        encarga de devolver la serialización del modelo en caso que el pedido
        sea ese, ya que el \code{proxyClient} sabe que cuando el cliente lo
        requiere, el \code{ProxyClient} necesita responder inmediatamente.
    \subsubsection{Módulo del modelo servidor}
        Este módulo está representado por la clase \code{Game} la cual
        interactúa con muchos \code{ProxyClient}, que son hilos. Por lo
        tanto esta clase es protegida. Está compuesta de métodos que definen las
        base del modelo del juego como la creación de unidades, el ataque a
        objetos y unidades y los movimientos de unidades. Además hay un método
        principal llamado \code{update()} el cual se encarga de actualizar el
        mismo recorriendo todas las unidades del juego, las municiones y los
        objetos y actualizando sus estados en caso de que estos necesiten serlo.
    \subsubsection{Modulo deserializacion del modelo}
        Este módulo está representado por la clase \code{ServerInterpreter} la
        cual se encarga de serializar el modelo servidor y almacenar dicha
        serialización en un vector. Además se encarga de desserializar los
        pedidos que llegan desde el cliente y llamar al método correspondiente
        del modelo.